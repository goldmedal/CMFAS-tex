\documentclass[CJK,12pt,t]{article}

% Any percent sign marks a comment to the end of the line

% Every latex document starts with a documentclass declaration like this
% The option dvips allows for graphics, 12pt is the font size, and article
%   is the style

%% 使用中文 (CJK package)
\usepackage{CJKutf8}
\usepackage[pdftex]{graphicx}
\usepackage{url}
\usepackage{array}
\usepackage{booktabs}
\usepackage{subfigure}
\usepackage{wrapfig}

\usepackage{amsmath}
\usepackage{amssymb}
\usepackage{amsthm}

% These are additional packages for "pdflatex", graphics, and to include
% hyperlinks inside a document.

\setlength{\oddsidemargin}{0.25in}
\setlength{\textwidth}{6.5in}
\setlength{\topmargin}{0in}
\setlength{\textheight}{8.5in}
\setlength{\parskip}{15pt}
\newcommand{\argmin}[1]{\underset{#1}{\operatorname{arg}\,\operatorname{min}}\;}
\renewcommand{\labelenumi}{\Alph{enumi})}

% These force using more of the margins that is the default style

\begin{document}
\begin{CJK*}{UTF8}{bkai}   %%% ZZZ %%%  <<< 在這裡更改預設中文字型、編碼
% Everything after this becomes content
% Replace the text between curly brackets with your own

%\title{}
%\author{}
%\date{}

% You can leave out "date" and it will be added automatically for today
% You can change the "\today" date to any text you like


%\maketitle

% This command causes the title to be created in the document

\section{主題四: 應用CMFAS架構於輪圈加工應用}
 
		在此主題我們會將CMFAS實際套用於輪圈加工應用,並且敘述CMFAS對於傳統雲端運算的改善。
		
		企業選擇將運算服務架設在雲端的原因在於,我們可以隨著不同的情境的計算量,去調整合適的機台規模,在成本與效能之間取得一個平衡。但在傳統雲端計算之中,最為人詬病的缺點便是資安問題。我們必須將己身的機密資料放置在服務供應商的機台之中,這導致我們對於己身機密資料的安全性無法掌握。

		\begin{figure}[ht]
			\begin{center}
				\includegraphics[scale=0.5]{figs/CASE2015.png}
				\caption{CMFAS架構圖}
				\label{cmfas1}
			\end{center}
		\end{figure}

		於是依照CMFAS的原則,我們提出了圖\ref{cmfas1}的架構,它解決了雲端系統帶來的資安問題,也兼顧了運算規模彈性調整的好處,以下做較詳盡之說明:
		\begin{itemize}
		\setlength{\itemsep}{5pt}
			\item 資料安全性:\\
					將具有機密性的資料建立在Manufacturing Database之中。並且建置在我們自行利用VMware所架設的私有雲之中,以此來確保我們對於機密資料的掌握度。\\
					在與IMF Processing Module資料交換之前,透過Authority Module來進行資料加密,來確保在公有雲進行的運算也是安全的。關於資料加密之方法,將於主題六會有更詳盡的解說。

			\item 運算規模調整:\\
				將需要高效能運算的IMF Processing Module建置在可以依照需求調整機台規模的公有雲(例如:Windows Azure或是 AWS EC2等),保持運算資源調配的彈性,方便我們在不同的情境底下,從運算效能與投資成本兩者之間取得一個平衡度。
		\end{itemize}

		我們以輪圈加工業為例,以主題一的CMFAS架構及上述原則來實現一個輪圈加工流程檢測系統。透過檢視NC檔的流程與刀具使用狀況,來預測是否會有碰撞或是刀具使用不當的行為發生,使用本體論之運算來進行刀具使用的修正,使得輪圈加工的流程得以順利進行。

		根據CMFAS框架,我們將本體論所需之知識庫與規則庫建立在Manufacturing Database之中,並將本體論運算與碰撞檢測實作於公有雲之IMF Processing Module中,搭配Cloud Service GUI來蒐集客戶端所送出之加工資訊,以此建構出WD-CAPP(Wheel Design Computer Aided process Planning)系統。

		\vspace{20pt}
		\begin{figure}[ht]
			\begin{center}
				\includegraphics[scale=0.299]{figs/step-wdcaspp.png}
				\caption{WD-CAPP運作流程圖}
				\label{cmfas2}
			\end{center}
		\end{figure}

		圖\ref{cmfas2}為整個WD-CAPP混和雲運作流程,以下為各流程說明

		\begin{enumerate}
		\setlength{\itemsep}{5pt}
			\item 從客戶端將所要使用之NC檔、輪圈加工資訊與欲使用之知識庫等加工資訊發送至Task Bulletin Board(TBB)
			\item 將所有加工資訊自Task Bulletin Board擷取至公有雲之Worker的BBC Module
			\item BBC Module根據加工資訊呼叫需要運作之Manufacturing Function。\\
					透過VMT Module來呼叫碰撞檢測之服務。若遇到刀具將會發生碰撞之情況,BBC Module將會呼叫Wheel Manf. Ontology Interency Module來根據知識庫與規則庫推薦正確的刀具。
			\item BBC Module將檢測結果與刀具推薦列表傳至Result Bulletin Board(RBB)
			\item 客戶端再將檢測結果與刀具推薦列表自Result Bulletin Board擷取下來。
		\end{enumerate}

		根據CMFAS這個框架所建立之加工流程檢測系統擁有極大的彈性調整空間,透過抽換不同的知識庫與規則庫便可以適用於不同地加工情境;調整Authority Module可以讓各個廠房擁有自己獨特的加密方式;最重要的則是根據不同情境的製造規模,來調整Worker數量及運算效能,達到資源最有效率地運用,讓整個加工產線的產能與利潤達到最佳化。
\end{CJK*}
\end{document}